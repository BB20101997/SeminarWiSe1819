%
% The first command in your LaTeX source must be the \documentclass command.
\documentclass[sigconf,nonacm,natbib=false]{acmart}
\settopmatter{printacmref=true}

\PassOptionsToPackage{ngerman}{babel}

\usepackage[style=apa,
citestyle=apa,
backend=biber,
sorting=none]{biblatex}
\addbibresource{references.bib}

%
% defining the \BibTeX command - from Oren Patashnik's original BibTeX documentation.
\def\BibTeX{{\rm B\kern-.05em{\sc i\kern-.025em b}\kern-.08emT\kern-.1667em\lower.7ex\hbox{E}\kern-.125emX}}
    
% Rights management information. 
% This information is sent to you when you complete the rights form.
% These commands have SAMPLE values in them; it is your responsibility as an author to replace
% the commands and values with those provided to you when you complete the rights form.
%
% These commands are for a PROCEEDINGS abstract or paper.
\copyrightyear{2018}
%\acmYear{2018}
%\setcopyright{acmlicensed}
%\acmConference[Woodstock '18]{Woodstock '18: ACM Symposium on Neural Gaze Detection}{June 03--05, 2018}{Woodstock, NY}
%\acmBooktitle{Woodstock '18: ACM Symposium on Neural Gaze Detection, June 03--05, 2018, Woodstock, NY}
%\acmPrice{15.00}
%\acmDOI{10.1145/1122445.1122456}
%\acmISBN{978-1-4503-9999-9/18/06}

%
% These commands are for a JOURNAL article.
%\setcopyright{acmcopyright}
%\acmJournal{TOG}
%\acmYear{2018}\acmVolume{37}\acmNumber{4}\acmArticle{111}\acmMonth{8}
%\acmDOI{10.1145/1122445.1122456}

%
% Submission ID. 
% Use this when submitting an article to a sponsored event. You'll receive a unique submission ID from the organizers
% of the event, and this ID should be used as the parameter to this command.
%\acmSubmissionID{123-A56-BU3}

%
% The majority of ACM publications use numbered citations and references. If you are preparing content for an event
% sponsored by ACM SIGGRAPH, you must use the "author year" style of citations and references. Uncommenting
% the next command will enable that style.
%\citestyle{acmauthoryear}

%
% end of the preamble, start of the body of the document source.
\begin{document}

%
% The "title" command has an optional parameter, allowing the author to define a "short title" to be used in page headers.
\title{Mögliche Probleme bei
der Anwendung von
Deep-Learning}
\subtitle{Poster Paper}

%
% The "author" command and its associated commands are used to define the authors and their affiliations.
% Of note is the shared affiliation of the first two authors, and the "authornote" and "authornotemark" commands
% used to denote shared contribution to the research.
\author{Bennet Bleßmann}
\authornote{Todo}
\email{bennet.blessmann@stu.uni-kiel.de}

%
% The abstract is a short summary of the work to be presented in the article.
\begin{abstract}
When applying Deep-Learning one has to lookout for some obvious and some hidden 
Problems which may occur in any of the development phases.
\end{abstract}

%
% The code below is generated by the tool at http://dl.acm.org/ccs.cfm.
% Please copy and paste the code instead of the example below.
%
\begin{CCSXML}
<ccs2012>
<concept>
<concept_id>10011007.10011074.10011092.10011782.10011813</concept_id>
<concept_desc>Software and its engineering~Genetic programming</concept_desc>
<concept_significance>500</concept_significance>
</concept>
<concept>
<concept_id>10011007.10011074.10011075.10011078</concept_id>
<concept_desc>Software and its engineering~Software design tradeoffs</concept_desc>
<concept_significance>300</concept_significance>
</concept>
<concept>
<concept_id>10011007.10011074.10011075.10011079</concept_id>
<concept_desc>Software and its engineering~Software implementation planning</concept_desc>
<concept_significance>300</concept_significance>
</concept>
<concept>
<concept_id>10011007.10011074.10011099.10011693</concept_id>
<concept_desc>Software and its engineering~Empirical software validation</concept_desc>
<concept_significance>300</concept_significance>
</concept>
<concept>
<concept_id>10002978</concept_id>
<concept_desc>Security and privacy</concept_desc>
<concept_significance>300</concept_significance>
</concept>
<concept>
<concept_id>10010147.10010257</concept_id>
<concept_desc>Computing methodologies~Machine learning</concept_desc>
<concept_significance>300</concept_significance>
</concept>
<concept>
<concept_id>10010520.10010521.10010542.10010294</concept_id>
<concept_desc>Computer systems organization~Neural networks</concept_desc>
<concept_significance>300</concept_significance>
</concept>
<concept>
<concept_id>10003120</concept_id>
<concept_desc>Human-centered computing</concept_desc>
<concept_significance>100</concept_significance>
</concept>
<concept>
<concept_id>10010405.10010455</concept_id>
<concept_desc>Applied computing~Law, social and behavioral sciences</concept_desc>
<concept_significance>100</concept_significance>
</concept>
</ccs2012>
\end{CCSXML}

\ccsdesc[500]{Software and its engineering~Genetic programming}
\ccsdesc[300]{Software and its engineering~Software design tradeoffs}
\ccsdesc[300]{Software and its engineering~Software implementation planning}
\ccsdesc[300]{Software and its engineering~Empirical software validation}
\ccsdesc[300]{Security and privacy}
\ccsdesc[300]{Computing methodologies~Machine learning}
\ccsdesc[300]{Computer systems organization~Neural networks}
\ccsdesc[100]{Human-centered computing}
\ccsdesc[100]{Applied computing~Law, social and behavioral sciences}

\ccsdesc[500]{Computer systems organization~Embedded systems}
\ccsdesc[300]{Computer systems organization~Redundancy}
\ccsdesc{Computer systems organization~Robotics}
\ccsdesc[100]{Networks~Network reliability}

%
% Keywords. The author(s) should pick words that accurately describe the work being
% presented. Separate the keywords with commas.
\keywords{datasets, neural networks, privacy, human-centered computing}

%
% A "teaser" image appears between the author and affiliation information and the body 
% of the document, and typically spans the page. 
%\begin{teaserfigure}
%  \includegraphics[width=\textwidth]{sampleteaser}
%  \caption{Seattle Mariners at Spring Training, 2010.}
%  \Description{Enjoying the baseball game from the third-base seats. %Ichiro Suzuki preparing to bat.}
%  \label{fig:teaser}
%\end{teaserfigure}


%
% This command processes the author and affiliation and title information and builds
% the first part of the formatted document.
\maketitle


\section{Introduction}

This paper to the poster attached in the appendix and will describe possible problems when applying deep-learning technologies and their possible effects. The problems will be grouped based on the development phase they are expect to be rooted in. Afterwards general solutions will be present were available otherwise instance specific solutions may be presented. Some problems may not be exclusive to deep-learning and this collection might not be complete. Reasons for assembling this collection include the increased use of automated decision making or in short ADM where deep-learning is a part of. In combination with more tools been available to make it easier to develop and deploy deep-learning applications it is important that the safety, security and reliability of deep-learning gets sufficiently analysed.
\section{Definitions}

\subsection{A Problem}
A Problem is anything which may result in unwanted negative effects of any kind.

\subsection{The Phases of Development}
For the Phases of development this paper will assume the following five stages similar to those found in the Waterfall model.
In real world application these phases might occur in parallel or multiple times or not at all.

The first Phase of Development is the phase of discovering and evaluating the Projects requirements, usually these might be constrained by a customers wishes and budget.

The second Phase of Development is the planning Phase where based on the result of the first Phase will be decided on the general structure of the Project.

The third Phase of Development is the Phase of actually implementing the Projects Plan and requirements.

The fourth Phase is where we have a larger divergent from the waterfall model as this is the Phase where the Algorithm will be trained, this is not a phase found in typical software development.

The last Phase is the Phase of Usage and Maintenance where the Product will be applied in the world and needs to be maintained.
\section{Problems}

\subsection{Requirements Phase}
Problems not being acknowledged in this phase might be hard to resolve at a later stage as this phase presents the foundation for all further phases. 

\subsubsection{Vague Goal}
%explanaition
For example an unspecified fairness metric would be a vague goal for a selection algorithm that could result in an algorithm to be considered unfair.
%example
Another example might be to have the goal to increase one's profit. This may be achieved by increasing efficiency to expand the profit margin on the product, or by implementing better advertisement strategies to get more buying consumers. Both would be valid strategies but, depending on the circumstances, either might be preferred over the other and each entails other possible problems. Therefore the goal to increase profit is vague as it is unclear which option is wanted.  

\subsubsection{Moral Acceptability}

%explanaition
Another less concrete problem is the consideration whether an algorithms should be developed at all in the sense that an algorithm can be technically possible but morally unwanted.

%example 
As an example:
I have no doubt that it is possible to develop better filters for the Great Firewall of China, but from a humanitarian standpoint I would disagree to build such.

\subsubsection{Economic Incentive}

%explanation
Probably the most tangible problem is the collision of morality and ethics with economics. 
In most circumstances, in  the current economic structure, it would be unreasonable to produce an algorithm without any direct economic benefit, even if it would be socially beneficial. 
On the other hand it might be of interest to produce an ethically, morally or socially problematic algorithm for one's economical benefit. 

%example
For example:
One could consider writing filters for China due to an economic benefit offered by the government.

\subsection{Planning Phase}
Since this is the phase in which we expect most decisions to be made and 
the most information to become available, this is also a rather important phase.

\subsubsection{Incorrect/Insufficient Model}
%explanation

A model might be insufficient by representing reality oversimplified or overcomplicated.

%example
For example the chatbot that had been deployed by Microsoft
 which turned into "[..] a racist asshole in less than a day" (\cite{James2018} was oversimplified, missing the difference between acceptable and unacceptable tweets.
 
\subsubsection{Insufficient Safety Considerations}
%explanation
This problem includes unconsidered IT security or data protection risks, as well as, where applicable, risk to the physical world.
External threats are usually sufficiently covered, while internal threats are sometimes forgotten or not sufficiently treated. 

\subsection{Implementation Phase}
This phase is basically identical to your typical implementation 
phase and therefore the same problems apply.

\subsubsection{Missing Documentation}
% Mangelnde Dokumentation (Annahmen, Funktionsweise etc.)

Missing documentation is mostly a problem in the case of undocumented assumptions which might result in unmet requirements and consequently in undefined behaviour. For example, if a function assumes inputs to be sanitised without this being documented, unsanitized data might be passed to it and lead to unwanted behaviour. Documentation can also help when maintaining a deployed product, which is part of a phase described later.

\subsection{Training Phase}
This is the last phase before the algorithm goes into production and also the one where the problems might be hard to predict, observe directly and diverge most from traditional programming. The focus will be on the data used for training rather than on the training itself.

\subsubsection{Restricted Representation of Reality}
% Unzureichende Abbildung der Realitzät durch Trainingsdaten

Training data is insufficient representative when it diverges from reality in essential parts. This can be seen at the example of election predictions (\cite{2016}) in the difference between the prediction and the result, where the difference is sometimes significant.

\subsubsection{Incorrect Information}
% Unreine Daten

Incorrect information might be misspelled names, mislabelled images et cetera. For example a misplaced comma in a report to a credit scoring firm might ruin one's credit score which has the potential to ruin a life.

\subsubsection{Irrelevant Information}
% Irrelevante Daten
%alg shall be used in amarica trained on world wide data

Irrelevant data is present when the data contains 
objects which are not supposed to be in the production 
environment. This does not mean the data is incorrect,
for example an algorithm only to be deployed to evaluate
Irish consumers that is also trained with data from 
Belgian consumers would fall into this category.
The problem is that in this example the algorithm
would have to handle more varied inputs with the same resources, resulting
in worse performance.

\subsubsection{Contaminated Data}
% Korrekte aber unpassende Daten (wiederspiegeln von Rassismus in der Gesselschaft etc.)

Additionally to the already mentioned cases, data can be problematic 
in at least one other way, as even correct, clean, truthfully representing data might not result in what was intended.
This can best be seen in the example of Amazon's employment evaluation algorithm (\cite{Higginbottom2018}). It was given the data of the currently employed workforce and was suppose to evaluate if applicants would be a good addition.
Since the workforce was dominantly male it was biased against females.

\subsection{Deployment/Maintenance Phase}
This phase mostly consists of problems inherent to
deep-learning algorithms. As such they are hardly 
able to be avoided.

\subsubsection{Lack of Traceability}
% Fehlende Nachvollziehbarkeit

In opposition to traditional algorithms, most deep-learning-algorithms are not comprehensible, as they are a set of numbers, computed based on some training data and an initial configuration, with no further justification, explanation or reason. In traditional algorithms an anomaly can usually be traced back to either an error in human intuition, meaning the result is correct even if this is not obvious, or an error in the algorithm which can than be resolved through correction. 

\subsubsection{Unforeseen Influences and Consequences}
% Unvorhergesehene Einflüsse

Unforeseen influences and consequences mostly come in pairs because one is usually the result of the other. For example it has been demonstrated (\cite{Eykholt2017})that it is possible to modify street signs in a way that an image classification algorithm misclassified street signs with high confidence, while being still legible from a human perspective.
%(Payed Positive Reviews -> Useless Reviews)

\subsubsection{Liability and Legal Responsibility}
% Unklare Haftung (Nutzer, Produzent, Entwickler ?)

As with all automated decision making processes 
it is not in all circumstances clear who is liable for 
the decisions made and has to take responsibility.
The example most talked about would probably be 
that of autonomous vehicles. Usually, either the driver or the manufacturer would be to blame for an accident, depending on the cause. In case of autonomous vehicles the car would be the main source of failure. This could cause a liability problem for manufacturers.   
Depending on the level of autonomy and the manufacturer complying with due diligence it might even be the case that neither the manufacturer nor the owner should be responsible. Which might pose a problem to the legal system, especially where a party exists that normally would have received compensation from the responsible party.

\section{Lösungen}
\subsection{Alte Daten}
\begin{frame}
\frametitle{Lösung}
\framesubtitle{Alte Daten}
\begin{itemize}[<+(1)->]
	\item Daten in einem Format Speichern welches nicht Veraltet, z.B. Geburtsjahr statt Alter
	\item Regelmäßig neue Daten sammeln und alte Daten löschen
\end{itemize}
%Alte Datensätze regelmäßig durch neue ersetzen
\end{frame}

\subsection{Keine/Zu wenig Daten}
\begin{frame}
\frametitle{Lösung}
\framesubtitle{Keine Daten/Zu wenig Daten}
\begin{itemize}[<+(1)->]
	\item Mehr Daten Sammeln
	\item Mehr Arten von Daten nutzen/sammeln
\end{itemize}
\end{frame}

\subsection{Verzerrte Daten}
\begin{frame}
\frametitle{Lösung}
\framesubtitle{Verzerrte Daten}
\begin{itemize}[<+(1)->]
\item Bei Credit-Scoring nicht nur Daten über negative Ereignisse sammeln sondern auch über positive Ereignisse 
\end{itemize}
%TODO ?
\end{frame}

\subsection{Falsche Daten}
\begin{frame}
\frametitle{Lösung}
\framesubtitle{Falsche Daten}
\begin{itemize}[<+(1)->]
	\item Einsicht\textbackslash Transparenz und Korrektur zulassen 
	\item Mit anderen Quellen vergleichen
	\item Fehlerquellen minimieren
\end{itemize}
%Daten einsicht, transparenz
%Daten prüfen und korrektur zulassen
\end{frame}

\subsection{Nicht repräsentative Daten}
\begin{frame}
\frametitle{Lösung}
\framesubtitle{Nicht repräsentative Daten}
\begin{itemize}[<+(1)->]
	\item Statt allgemeine Daten zu nutzen, speziell für einen Zweck Daten Sammeln
	\item Maßnahmen gegen fehlende, zu wenige und verzerrte Daten
\end{itemize}
%Daten für den Zweck nutzen für die sie Gesammelt wurden 
%Daten sammeln welche zum Zweck passen
\end{frame}

\subsection{Zu viele Daten}
\begin{frame}
\frametitle{Lösung}
\framesubtitle{Zu viele Daten}
%Daten löschen welche verzichtbar/weniger relevant sind
\end{frame}
\section{Summary}

The problems of the requirement phase and implementation phase are basically equivalent to their respective phases for development in general. As such, similar solutions apply. The other phases differ to a greater extend. In traditional programming it is easier to find a working model in the planning phase, as too much data is seldom a problem and can simply be unused, but as with the Amazon example (\cite{Higginbottom2018}), too much data in deep-learning applications can easily lead to unwanted results and may even be hard to remove.
The training phase is not found in typical programming although it bears similarities to testing. Again the Amazon example demonstrates how subtle contaminated data may be. While incorrect information might not be detectable at all, an error in the data can still represent a potentially valid entry. The deployment/maintenance phase also exists in the traditional development scenario and shares some similar problems, mainly unforeseen influences and consequences. Though it manifests itself differently, in the traditional logic errors would be exploited and would in general be limited to have consequences outside of the physical world,
while attacks on deep-learning algorithms, like in the example of sign manipulation,
might be an inherent problem with this approach, which has the potential to cause incidence in the physical world. 
%Last %TODO mention 3.5.3

\section{Conclusion}
As with most technologies, deep-learning is not a one fits all solution for all problems. Deep-Learning is an evolving technology with potential, but one should not overlook the problems that are still unsolved and needs to lookout for those we already can solve. The problems shared with development in general are known and we manage to work with them already. Therefore this is where I see the least problem for the future. The part with the worst outlook would be the lack of traceability. I am not optimistic that this will be solvable, but it would be something worth investigating, should a promising lead be discovered. In my opinion, this is a major drawback of the deep-learning approach. As for the liability and legal responsibility, we will probably need to wait for the first incidents to occur, where this plays a significant role. All remaining problems are different on a case by case basis and probably can't be solved in general. But as with semantic errors in software development, I would assume that it should be possible to eliminate some of the most common errors by providing tooling and development approaches that aid the correct development and implementation of deep-learning applications.


%
% The acknowledgments section is defined using the "acks" environment (and NOT an unnumbered section). This ensures
% the proper identification of the section in the article metadata, and the consistent spelling of the heading.
\begin{acks}
To Daniela Stollberg for helping to check the Paper for errors.
\end{acks}

%
% The next two lines define the bibliography style to be used, and the bibliography file.

\nocite{*}

\printbibliography

% 
% If your work has an appendix, this is the place to put it.
\appendix

\end{document}
