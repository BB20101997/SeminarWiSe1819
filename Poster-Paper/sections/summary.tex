\section{Summary}

The problems of the requirement phase and implementation phase are basically equivalent to their respective phases for development in general, as such similar solutions apply. The other phases differ to a greater extend. In traditional programming it is easier to find a working model, in the planning phase, as too much data is seldom a problem and can simply be unused, but as with the Amazon example (\cite{Higginbottom2018}) too much data in deep-learning applications can easily lead to unwanted results and may even be hard to remove.
The training phase is not found in typical programming thou it bears similarities to testing. Again the Amazon example demonstrates how subtle contaminated data may be, while dirty data might not be detectable at all as an error in the data might still represent a potentially valid entry. The deployment/maintenance phase also exists in the traditional development scenario and shares some similar problems mainly Unforeseen Influences and Consequences, thou it manifests itself differently,in the traditional logic errors would be exploited and would in general be limited to have consequences outside of the physical world,
while attacks on deep-learning algorithms with the example of sign manipulation
might be an inherent problem with this approach which has the potential to cause incidence in the physical world. Last 
%TODO mention 3.5.3

\section{Conclusion}
As with most technologies deep-learning is not a one fits all solution for all problems. Deep-Learning is an evolving technology with potential, but one should not overlook the problems that are still unsolved and needs to lookout for those we already can solve. The problems shared with development in general are known and we manage to work with them already therefore this is where I see the least problem for the future. The part with the most grim outlook would be the lack of traceability I am not optimistic that this will be solvable but this would be something worth investigating should a promising lead be discovered, as this is a major drawback of the deep-learning approach in my opinion. As for the Liability and Legal Responsibility we will probably need to wait for the first incidents to occur where this plays a significant role. All remaining problems are different on a case by case basis and probably can't be solved in generally, but as with semantic errors in software development I would assume that it should be possible to eliminate some of the most common errors by providing tooling and development approaches that aid the correct development and implementation of deep-learning applications.
