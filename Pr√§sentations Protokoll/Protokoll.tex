\documentclass[12pt,a4paper]{article}
\usepackage[utf8]{inputenc}
\usepackage[ngerman]{babel}
\usepackage{amsmath}
\usepackage{amsfonts}
\usepackage{amssymb}
\author{Bennet Bleßmann}
\title{Protokoll: Einfluss der Datenbasis}
\begin{document}
\maketitle

\subsection*{Präsentation}
	Der Begriff Datenbasis schien nicht so klar zu sein wie erwartet,
	dies schien aber geklärt worden zu sein.
\subsection*{Was wäre wenn?}
	Anfangs befasste sich die Diskussion damit,
	dass es an Informationen fehlt welche Daten die Datenbasis ausmachen.
	Daraufhin wurde überlegt welche Informationen sinnvoll und vorstellbar wären und
	welche impliktionen diese haben könnten.
	Hierbei wurde unter anderem gennannt das die Familiäre herkunft 
	mit einbezogen werden könnte , so dass dann Personen welche von Personen abstammen,
	welche bereits Studiert hätten somit wohl bevorzugt würden.
	Weitere Faktoren wahren der Familiäre wohlstand sowie die Familien Konstellation
	also z.B.	Einzelkind,
				Kind adoptiert in einer nicht Heterosexuelle Ehe,
				Kind alleinerziehend großghezogen etc.
				 
	Weiter wurde angemerkt,
	 dass das eingene Interresse einer Person untergehen könnete wenn auf grund z.B.
	 Schulischer Leistungen oder Proffession von z.B. Eltern Studiengänge empfohlen würden.
	
	Es wurde auch Diskutiert,
	dass das Ergebniss des Algorithmuss keien Empfehlung sein könnte
	sondern eine Einschränkung sein könnte.
		
\subsection*{Alltags Beispiel}
	Die Diskussion hier viel unerwartet kurz aus es kam nur ein Beispiel aus dem Publikum 
	nähmlich die Vervollständundsvorschläge bei z.B. Google.
	Darauf hin hab ich noch die mir Beispiele gennent welche mir noch einfielen, also Autokorrektur im Handy und personallisierte Suchergebnisse.

\subsection*{Szenario}

	Es wurde über den geringen Umfang an Daten und den (daher) fehlenden Kontext diskutiert,
	für alle drei erwarteten Richtungen haben sich Argumente gefunden, es tendierte hierbei dazu Einzugreifen, da die möglichkeit eines Anschlags gesehen wurde und im falle das dies das Fall wäre könnte dieser verhinder werden und im falle das dies eine Fehleinschätzung wäre seien die Auswirkungen recht gering für die allgemeinheit wenn auch möglicherweise gravierend für die einzel Person. Auch das nichts Unternahemen wurde verteidigt mit der Begründung, das zu den Bekannten Informationen viel Kontext fehlt und das die Resourcen was Spezialkräfte anbelangt begrenzt sind und man davon ausgehen müsse das bei einem so geringen Datensatz viele Menschen verdächtig wirken werden.
	
	Nachdem die "Neuen Fakten" mitgeteilt wurde, wurde angemerkt das die neuen Informationen die Ursprünglichen in einen neuen Kontext rücken welcher eher für das nicht eingreifen spricht, unerwarteter weise gab es auch den berechtigten Einwand das auch wenn jetzt die ersten Informationen einen Kontext haben in dem sie unschuldig wirken sie es nicht auch sein müssen, da sich ja  auch ein Student der Chemie mit NF Gesellschaftswissenschaften der Angelt und Heimhandwerkelt radikalisieren kann.
	
\subsection*{TLDR}

	Der erste Diskussions Teil ist Anfangs unerwartet allgemein geblieben, gegen Ende aber auch bei Persönlichen Beispielen, Erfahrungen und Erwartungen angelangt, es wurde ersichtlich das Algorithmen und ihre Implikationen hinterfragt wurden und dem Publikum bewusst sind.
	
	Im zweiten Teil der Diskussion wurde deutlich das die Datenbasis und ihre Implikatiinen noch nicht bewusst wargenommen werden, da hier kaum Beispiele gennant werden konnten.
	
	In der ersten hälfte des dritten Teils wurde wie auch im ersten Teil erst allgemein der geringe/fehlende Kontext besprochen um dann die Entscheidungen zu Begründen, welche zum größeren Teil für ein Eingreifen waren.
	In der zweiten hälfte des letzten Teils als die Weiteren Informationen bekannt waren, war am Interresatntesten das auch wenn der neue Kontext wohl für kein Eingreifen präferiert war, es doch einen noch größeren Kontext geben konnte in dem ein Eingreifen dennoch nötig wäre.
	
\end{document}