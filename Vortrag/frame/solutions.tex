\section{Lösungen}
\subsection{Alte Daten}
\begin{frame}
\frametitle{Lösung}
\framesubtitle{Alte Daten}
\begin{itemize}[<+(1)->]
	\item Daten in einem Format Speichern welches nicht Veraltet, z.B. Geburtsjahr statt Alter
	\item Regelmäßig neue Daten sammeln und alte Daten löschen
\end{itemize}
%Alte Datensätze regelmäßig durch neue ersetzen
\end{frame}

\subsection{Keine/Zu wenig Daten}
\begin{frame}
\frametitle{Lösung}
\framesubtitle{Keine Daten/Zu wenig Daten}
\begin{itemize}[<+(1)->]
	\item Mehr Daten Sammeln
	\item Mehr Arten von Daten nutzen/sammeln
\end{itemize}
\end{frame}

\subsection{Verzerrte Daten}
\begin{frame}
\frametitle{Lösung}
\framesubtitle{Verzerrte Daten}
\begin{itemize}[<+(1)->]
\item Bei Credit-Scoring nicht nur Daten über negative Ereignisse sammeln sondern auch über positive Ereignisse 
\end{itemize}
%TODO ?
\end{frame}

\subsection{Falsche Daten}
\begin{frame}
\frametitle{Lösung}
\framesubtitle{Falsche Daten}
\begin{itemize}[<+(1)->]
	\item Einsicht\textbackslash Transparenz und Korrektur zulassen 
	\item Mit anderen Quellen vergleichen
	\item Fehlerquellen minimieren
\end{itemize}
%Daten einsicht, transparenz
%Daten prüfen und korrektur zulassen
\end{frame}

\subsection{Nicht repräsentative Daten}
\begin{frame}
\frametitle{Lösung}
\framesubtitle{Nicht repräsentative Daten}
\begin{itemize}[<+(1)->]
	\item Statt allgemeine Daten zu nutzen, speziell für einen Zweck Daten Sammeln
	\item Maßnahmen gegen fehlende, zu wenige und verzerrte Daten
\end{itemize}
%Daten für den Zweck nutzen für die sie Gesammelt wurden 
%Daten sammeln welche zum Zweck passen
\end{frame}

\subsection{Zu viele Daten}
\begin{frame}
\frametitle{Lösung}
\framesubtitle{Zu viele Daten}
%Daten löschen welche verzichtbar/weniger relevant sind
\end{frame}