\section{Diskussion}

\begin{frame}
\frametitle{Diskussion}
\setbeamercovered{transparent}

\begin{itemize}
\item Was wäre wenn?
\item Alltags Beispiele
\item Szenario
\end{itemize}

\end{frame}

\begin{frame}
\subsection*{Was wäre wenn?}
\frametitle{Diskussion}
\framesubtitle{Was wäre wenn?}
\setbeamercovered{transparent}
Angenommen die Zulassung und die Auswahl des Studienganges würde von einem Algorithmus abhängen würde.

Wer von den Anwesenden könnte sich denken heute nicht hier zu sein (und warum)? 

\end{frame}

\begin{frame}
\frametitle{Diskussion}
\framesubtitle{Alltags Beispiele}
\setbeamercovered{transparent}
\subsection*{Alltags Beispiele}
\only<handout>{
	Beispiel: Navi passt aufgrund einer \textit{während der fahrt} mitgeteilten Baustelle die Strecke automatisch an und Informiert über die Anpassung.
}
\only<beamer:2|handout:0>{
	Warum ist dies ein Beispiel für den Einfluss der Datenbasis?
}
\only<beamer:3|handout:0>{
	Kennt jemand hier ein vergleichbares Beispiel?
}
\end{frame}

\begin{frame}
\frametitle{Diskussion}
\framesubtitle{Szenario}
\setbeamercovered{transparent}
\subsection*{Szenario}

\only<beamer:1>{
	Einführung
}
\only<handout:1|beamer:0>{
	Im rahmen der Vorrats Datenspeicherung sind sie ein vom BND eingesetzter Algorithmus zur Evaluierung der eingehenden Daten. Ihre Aufgabe ist es basierend auf den Vorhandenen Daten z.B. Einsätze zu initiieren, Nichts zu unternehmen oder an einen Menschen weiterzuleiten.
}

\only<2>{
	\begin{itemize}
		\item Besuchen von vielen Seiten zum Thema \textbf{ISIS}
		\item Besuchen von vielen Seiten zu \textbf{Explosiven Gemischen}
		\item Kauf einer Angelweste
		\item Kauf von Rohren
		\item Kauf verschiedener Haushaltsmittel
		\item Last-Minute Kauf von Karten für einen großen Kongress
	\end{itemize}
}

\only<beamer:3|handout:0>{
	Was soll der Algorithmus entscheiden?
}

\only<beamer:4|handout:0>{
	\begin{itemize}
		\item Nichts Unternehmen, keine Gefahr ausgehend!
		\item Kavallerie zum aufhalten schicken, Bombenanschlag wahrscheinlich!
		\item Eskalieren und einen Menschen entscheiden lassen!
	\end{itemize}
}

\only<beamer:5|handout:0>{
	Neue Fakten:
	
	\begin{itemize}
		\item Studiert Chemie mit Nebenfach Gesellschaftswissenschaftlich
		\item Hat Angeln als Hobby
		\item Hat defektes Rohr selbst ersetzt statt den Handwerker zu rufen.
	\end{itemize}
}

\only<beamer:6|handout:0>{
	Reflexion:
	
	\begin{itemize}
		\item War die ursprüngliche Entscheidung korrekt?
		\item Würde eine neuen Entscheidung anders ausfallen?
		\item Hatten wir zu wenige oder zu viele Daten?
	\end{itemize}
}

\end{frame}